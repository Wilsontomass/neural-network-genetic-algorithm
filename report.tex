\documentclass[a4paper, 11pt, twocolumn]{report}

\usepackage[swedish]{babel}
\usepackage[T1]{fontenc}
\usepackage[utf8]{inputenc}
\usepackage{amsmath}
\usepackage{graphicx}
\usepackage{fancyhdr}
\usepackage{csquotes}

\usepackage[style=authoryear-ibid,backend=biber]{biblatex}

\addbibresource{sources.bib}% Syntax for version >= 1.2

\pagestyle{fancy}

\lhead{Tomass Wilson}
\rhead{19991030-0616}
\rfoot{\thepage}
\renewcommand{\headrulewidth}{0.4pt}
\renewcommand{\footrulewidth}{0.4pt}

\title{Evolutionära Neuronala Nätverk, ett Effektivitetsstudie}
\author{Tomass Wilson\\thmwi@kth.se}

\begin{document}

  \maketitle

  \begin{abstract}
    Sammanfattning
  \end{abstract}

  \tableofcontents

  \section{inledning}
    Programmerare har alltid velat lösa problem, hellre med hjälp av en dator. När Artificiella Neuronala Nätverk (ANN) började utvecklas kunde man applicera dem på problem som före-detta verkade omöjliga, såsom att urskilja ansikter, eller kategorisera bilder \cite{hopfield1988artificial}. 

\printbibliography

\end{document}
