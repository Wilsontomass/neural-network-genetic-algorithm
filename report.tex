\documentclass[a4paper, 11pt, twocolumn]{article}

\usepackage[swedish]{babel}
\usepackage[T1]{fontenc}
\usepackage[utf8]{inputenc}
\usepackage{amsmath}
\usepackage{graphicx}
\usepackage{fancyhdr}
\usepackage{csquotes}

\usepackage[style=authoryear-ibid,backend=biber]{biblatex}

\addbibresource{sources.bib}% Syntax for version >= 1.2

\pagestyle{fancy}

\lhead{Tomass Wilson}
\rhead{Grupp B}
\rfoot{\thepage}
\renewcommand{\headrulewidth}{0.4pt}
\renewcommand{\footrulewidth}{0.4pt}

\title{Evolutionära Neuronala Nätverk, ett Effektivitetsstudie}
\author{Tomass Wilson\\thmwi@kth.se}

\begin{document}

  \maketitle

  \begin{abstract}
    Sammanfattning
  \end{abstract}

  \tableofcontents

  \section{Inledning}
    Programmerare har alltid velat lösa problem, hellre med hjälp av en dator. När Artificiella Neurala Nätverk (ANN) började utvecklas kunde man applicera dem på problem som före-detta verkade omöjliga, såsom att urskilja ansikten eller kategorisera bilder \parencite{hopfield1988artificial}. Det blir en allt större och viktigare del av moderna program och då finns det stor anledning till att försöka korta ner tiden det tar för att bygga ett sådant nätverk. Inlärningsprocessen tar ofta flera timmar och bygger på att slumpa fram de kopplingarna mellan noder (neuroner) sådan att en viss korrekthet på resultatet nås för det specificerade problemet. Effektiviteten för ANN brukar definieras vid 2 variabler; inlärningstid och korrekthet, och detta mäts med hjälp av standardiserade tester. En viktig faktor som påverkar inlärningen av en ANN är meta-variabler, såsom antal noder per lager, antal lager, aktiveringsfunktionen, etcetera. Dessa meta-variabler kan ställas in manuellt, men detta brukar leda till att nätverket inte når sitt maximala precision, eller förlänger inlärningstiden märkbart. Lösningen till detta har tidigare varit att iterera genom alla möjliga combinationer meta-variabler och testa dem var-för-sig. En föreslagen lösning till detta är att istället evolutionärt optimera slumpmässigt valda meta-variabler för att undvika onödiga tester av ineffektiva kombinationer av dessa.

    \subsection{Syfte}
    Syftet med denna studie är att undersöka om tiden för inlärandeprocessen för ett neuralt nätverk kan förkortas med hjälp av en evolutionär process. Precisionen för nätverket ska vara helst den samma som vid en iterativ inlärningsprocess, för att kunna ge en komparabel alternativ.
    \subsection{Frågeställning}
    Vilken inställning av Matt Harveys neurala nätverk har kortast inlärningstid?

  \section{Bakgrund \& Teori}
    \subsection{Artificiella Neurala Nätverk}
    En Artificiell Neural Nätverk (ANN) är en samling sammankopplade noder sammanställda i flera lager. Varje nod har ett värde och en aktiveringsfunktion. En nod har flera inkommande länkar och utgående länkar till andra noder. Inkommande värden multipliceras med en slumpmässig vikt och sedan multipliceras alla inkommande värden tillsammans. Detta nya värde kläms sedan ner till ett värde mellan 0 och 1. Denna process kallas \textit{aktiveringsfunktionen} och slutgiltiga resultatet är nodens \textit{värde}. Antal noder i input- och outputlagern bestäms i förhand beroende på hur nätverket ska användas.

    \subsection{Meta-variabler}
    För att ett ANN ska kunna skapas behövs några parametrar. Dessa \textit{meta-variabler} bestämmer övergripande hur nätverket ser ut, dess storlek och konstruktion.

    \subsection{Evolutionär Inlärningsprocess}

    \subsection{CIFAR-10}
    Industri-standarden för att testa ANN är CIFAR-10, en databas av 60 000 bilder, som kategoriseras in i 10 grupper \parencite{krizhevsky2014cifar}. Med detta kan man okomplicerat jämföra olika ANN, genom att mäta hur snabbt och väl de kan identifiera vilken kategori en bild tillhör. Som input får nätverket en bild av 32 x 32 pixlar, som mäts in i Input lagern som en matrix. Output lagern har 10 noder, med ett värde mellan 0 och 1 för att visa hur mycket bilden ”passar” in i en viss kategori. Korrektheten mäts som andel gånger nätverket har placerat bilden i korrekt kategori

  \section{Metod}
  För att jämföra de två olika metoder för att bestämma bästa meta-variabler använder vi av CIFAR-10 databasen för materialet. Själva processen av populationskapande, inlärning, mätning och evolution är skriven av Matt Harvey \parencite{harvey2017}. All kod skrivs i python 3.6, och använder sig till huvuddel av tqdm, keras och tensorflow paketen för att optimera och skapa nätverken. Versionerna av samtliga paket ses i tabell \ref{paketversioner}.

  \begin{table}
    \centering
    \begin{tabular}{c|c}
      Paket & Version \\
      \hline
      keras & 2.2.2 \\
      keras-applications & 1.0.4 \\
      keras-preprocessing & 1.0.2 \\
      numpy & 1.15.1 \\
      scipy & 1.1.0 \\
      six & 1.11.0 \\
      tensorflow & 1.10.1 \\
      tqdm & 4.25.0 \\
      werkzeug & 0.14.1 \\
      wheel & 0.31.1
    \end{tabular}
    \caption{Paketversioner}
    \label{paketversioner}
  \end{table}

  \section{Resultat}

  \section{Diskussion}


\printbibliography

\end{document}
